\section{Methods}
\label{sec:methods}

\subsection{Experimental Data}

Our analysis employs two distinct categories of quantum measurement data to validate MQC analysis as a practical diagnostic tool:

\subsubsection{Bell State Data}
We utilize measurement data from four maximally entangled Bell states:
\begin{align}
|\Phi^+\rangle &= \frac{1}{\sqrt{2}}(|00\rangle + |11\rangle) \\
|\Phi^-\rangle &= \frac{1}{\sqrt{2}}(|00\rangle - |11\rangle) \\
|\Psi^+\rangle &= \frac{1}{\sqrt{2}}(|01\rangle + |10\rangle) \\
|\Psi^-\rangle &= \frac{1}{\sqrt{2}}(|01\rangle - |10\rangle)
\end{align}

Each Bell state was measured 8,192 times, providing high-statistics data for analysis. These states serve as clean reference data due to their theoretical purity and well-characterized properties.

\subsubsection{IBM Quantum Volume Data}
We analyze real experimental data from IBM Quantum Volume (QV) experiments, which represent realistic noisy quantum computations. The dataset includes:
\begin{itemize}
\item Standard QV data: 70 experimental trials
\item Moderate noise QV data: 100 trials with intermediate noise levels  
\item High noise QV data: Variable trials with elevated noise conditions
\end{itemize}

These datasets provide realistic examples of quantum computations affected by various noise sources including decoherence, gate errors, and measurement errors.

\subsection{Trajectory Generation}

For each dataset, we generate complex trajectories using the MQC representation described in Section~\ref{sec:theory}. The specific implementation details are:

\subsubsection{Preprocessing}
\begin{enumerate}
\item Raw measurement counts are converted to binary sequences
\item Sequences are subsampled to ensure computational efficiency while preserving statistical properties
\item Window sizes are chosen adaptively based on sequence length: $w = \min(50, N/10)$
\end{enumerate}

\subsubsection{Complex Mapping}
Each binary measurement sequence $\{m_i\}$ is transformed into a complex trajectory $\{z_i\}$ using Equations~\ref{eq:direction} and \ref{eq:uncertainty}, with physical constraints applied according to Equations~\ref{eq:constraint_real} and \ref{eq:constraint_imag}.

\subsection{Analysis Pipeline}

Our comprehensive analysis pipeline consists of four main components:

\subsubsection{Geometric Analysis}
For each trajectory, we compute:
\begin{itemize}
\item Instantaneous properties: velocity (Eq.~\ref{eq:velocity}), acceleration (Eq.~\ref{eq:acceleration}), curvature (Eq.~\ref{eq:curvature})
\item Global properties: tortuosity (Eq.~\ref{eq:tortuosity}), total path length, trajectory area
\item Topological properties: winding number (Eq.~\ref{eq:winding}), self-intersection count
\end{itemize}

\subsubsection{W-Pattern Feature Extraction}
We implement specialized algorithms to detect and quantify W-shaped patterns in trajectories:
\begin{itemize}
\item Fractal dimension estimation using box-counting methods
\item Detection of characteristic turning points and inflection points
\item Quantification of pattern regularity and symmetry
\end{itemize}

\subsubsection{Spectral Analysis}
Frequency domain analysis is performed using:
\begin{itemize}
\item Fast Fourier Transform (FFT) of complex trajectories
\item Power spectral density computation for real and imaginary components
\item Spectral entropy calculation (Eq.~\ref{eq:spectral_entropy})
\item Time-frequency analysis using spectrograms
\end{itemize}

\subsubsection{Statistical Comparison}
We implement robust statistical methods for comparing trajectory properties:
\begin{itemize}
\item Kolmogorov-Smirnov tests for distribution comparisons
\item Bootstrap resampling for confidence interval estimation
\item Multi-dimensional scaling for trajectory clustering
\end{itemize}

\subsection{Error Detection Implementation}

The error detection system operates in two phases, incorporating specialized detection algorithms for different error types based on trajectory pattern analysis:

\subsubsection{Training Phase}
\begin{enumerate}
\item Reference statistics are computed from clean Bell state trajectories
\item Threshold values are determined using 3-sigma criteria on reference distributions
\item Feature weights $w_p$ in Equation~\ref{eq:error_metric} are optimized using cross-validation
\end{enumerate}

\subsubsection{Detection Phase}
\begin{enumerate}
\item Test trajectories are processed through the same analysis pipeline
\item Error metrics are computed using Equation~\ref{eq:error_metric}
\item Binary classification is performed based on threshold comparison
\item Confidence scores are assigned based on the magnitude of deviation from reference
\end{enumerate}

\subsubsection{Specialized Error Type Detection}
Building on trajectory pattern recognition, we implement targeted detection algorithms:

\begin{itemize}
\item \textbf{Bit-flip Error Detection}: Monitoring trajectory curvature spikes and sudden direction changes
\item \textbf{Phase Noise Detection}: Analyzing trajectory wandering patterns and phase drift
\item \textbf{Amplitude Decay Detection}: Tracking systematic drift toward the complex plane origin
\item \textbf{Coherence Loss Detection}: Measuring trajectory regularity and pattern degradation
\end{itemize}

Each detection algorithm uses error-specific thresholds optimized for maximum sensitivity while maintaining low false positive rates.

\subsection{Computational Implementation}

All analyses are implemented in Python 3.8+ using:
\begin{itemize}
\item NumPy and SciPy for numerical computations
\item Matplotlib for visualization
\item Custom algorithms for MQC-specific calculations
\end{itemize}

The complete analysis pipeline processes datasets containing thousands of measurements within minutes on standard desktop hardware, making it suitable for real-time quantum device monitoring applications.

\subsection{Validation Methodology}

To ensure robustness of our results, we employ:
\begin{itemize}
\item Cross-validation using different subsets of data
\item Sensitivity analysis for parameter variations
\item Comparison with traditional quantum state analysis methods
\item Statistical significance testing for all reported differences
\end{itemize}