\documentclass[aps,pra,twocolumn,showpacs,preprintnumbers,amsmath,amssymb]{revtex4-1}

\usepackage{graphicx}
\usepackage{amsmath}
\usepackage{amssymb}
\usepackage{bm}
\usepackage{color}
\usepackage{hyperref}
\usepackage{subcaption}

\begin{document}

\preprint{APS/123-QED}

\title{Measurement Quality Complex (MQC) Analysis: A Visual Tool for NISQ-Era Quantum Computing}

\author{Keisuke Otani}
\email{wacon316@gmail.com}

\date{\today}

\begin{abstract}
We present a practical visualization and analysis tool for quantum measurement data using Measurement Quality Complex (MQC) representation. Our method achieves noise-dependent error detection rates ranging from 8.5\% to 26.4\% on IBM Quantum Volume experiments while maintaining computational efficiency suitable for real-time monitoring. The MQC approach represents measurement sequences as complex numbers $z = d + ui$ where $d$ is directional bias and $u$ is uncertainty, providing intuitive visualization of quantum device performance. Testing on Bell states and IBM Quantum Volume data demonstrates quantitative discrimination with 82-fold tortuosity difference and 3.6-fold spectral entropy variation between clean and noisy measurements. This tool offers NISQ-era quantum computing practitioners a complementary diagnostic method for hardware characterization and real-time quality assessment.
\end{abstract}

\keywords{NISQ computing, quantum diagnostics, measurement analysis, real-time monitoring, quantum hardware characterization}

\maketitle

\section{Introduction}

NISQ-era quantum devices face significant challenges from noise and decoherence that require real-time monitoring and diagnostic tools~\cite{preskill2018quantum}. Traditional quantum error analysis relies on computationally intensive methods like quantum process tomography or provides limited information like randomized benchmarking~\cite{nielsen2010quantum}. There is a critical need for practical tools that can provide visual, intuitive feedback about quantum device performance while maintaining computational efficiency suitable for real-time applications.

Current diagnostic approaches have significant limitations: quantum process tomography scales exponentially with system size, making it impractical for real-time monitoring, while randomized benchmarking provides only averaged fidelity information without detailed error characterization~\cite{arute2019quantum,zhong2020quantum}. Hardware engineers and quantum algorithm developers need complementary tools that can quickly identify performance degradation and provide actionable insights.

We introduce Measurement Quality Complex (MQC) representation, a practical analysis tool for quantum measurement data. MQC numbers are defined as:
\begin{equation}
z = d + ui
\end{equation}
where $d$ represents the directional bias of measurements and $u$ represents the uncertainty.

\textbf{Important}: These complex numbers are \emph{not} quantum amplitudes. They represent statistical properties of measurement sequences, not quantum states. MQC analysis transforms binary measurement outcomes into visual trajectories in the complex plane, enabling intuitive understanding of quantum device behavior patterns.

Our contributions include:
\begin{itemize}
\item A computationally efficient visualization tool for NISQ device characterization
\item Real-time capable error detection with quantitative performance metrics
\item Intuitive visual representation complementing existing diagnostic methods
\item Validation on real quantum hardware data with measurable discrimination capability
\end{itemize}

We validate MQC analysis using real quantum data from Bell state measurements and IBM Quantum Volume experiments. Our results demonstrate quantitative discrimination capability with 82-fold difference in trajectory tortuosity and 3.6-fold difference in spectral entropy between clean and noisy measurements, achieving practical error detection suitable for real-time monitoring applications.

The remainder of this paper presents the MQC representation framework, validation methodology using real quantum data, quantitative performance results, and discussion of practical applications for NISQ-era quantum computing diagnostics.
\section{Measurement Quality Complex Representation}
\label{sec:theory}

\subsection{Definition and Motivation}

We define MQC not as a fundamental physical quantity, but as a useful statistical representation for visualizing quantum measurement quality. For a sequence of quantum measurements yielding outcomes $m_i \in \{0, 1\}$, we construct complex numbers:

\begin{equation}
z_i = d_i + i u_i
\label{eq:cqt_basic}
\end{equation}

where $d_i$ represents the directional bias and $u_i$ represents the measurement uncertainty within a local window.

\subsection{Distinction from Quantum Amplitudes}

\begin{table}[h]
\centering
\small
\begin{tabular}{lll}
\hline
Property & Quantum Amplitude & MQC \\
\hline
Physical meaning & Probability amplitude & Statistical indicator \\
Normalization & $\sum|\psi|^2 = 1$ & Not required \\
Interference & Yes & No \\
\hline
\end{tabular}
\caption{Fundamental differences between quantum amplitudes and MQC numbers.}
\end{table}

MQC numbers represent patterns in measurement data, not quantum mechanical amplitudes. This distinction is crucial for proper interpretation of results.

\subsection{Direction and Uncertainty Mapping}

The directional component $d_i$ is computed as a running correlation measure:

\begin{equation}
d_i = \frac{1}{w} \sum_{j=i-w+1}^{i} \frac{2m_j - 1}{\sqrt{1 + \sigma^2_j}}
\label{eq:direction}
\end{equation}

where $w$ is the window size for local averaging and $\sigma^2_j$ is the local variance of measurements.

The uncertainty component $u_i$ captures the unpredictability in the measurement sequence:

\begin{equation}
u_i = \sqrt{\frac{1}{w-1} \sum_{j=i-w+1}^{i} (m_j - \bar{m}_i)^2}
\label{eq:uncertainty}
\end{equation}

where $\bar{m}_i$ is the local mean of measurements within the window.

\subsection{Physical Constraints}

To ensure physical consistency, we enforce the constraints:
\begin{align}
d_i &\in [-1, +1] \label{eq:constraint_real}\\
u_i &\in [0, 1] \label{eq:constraint_imag}
\end{align}

These constraints ensure that the real part represents a normalized correlation measure and the imaginary part represents a normalized uncertainty measure.

\subsection{Trajectory Construction}

A sequence of $N$ measurements generates a complex trajectory:

\begin{equation}
\mathcal{T} = \{z_1, z_2, \ldots, z_N\}
\label{eq:trajectory}
\end{equation}

This trajectory can be analyzed using various mathematical tools:

\subsubsection{Geometric Properties}

The instantaneous velocity of the trajectory is given by:
\begin{equation}
v_i = |z_{i+1} - z_i|
\label{eq:velocity}
\end{equation}

The acceleration is:
\begin{equation}
a_i = |v_{i+1} - v_i|
\label{eq:acceleration}
\end{equation}

The curvature at point $i$ is computed as:
\begin{equation}
\kappa_i = \frac{|\text{Im}[(z_{i+1} - z_i)\overline{(z_i - z_{i-1})}]|}{|z_{i+1} - z_i|^2 |z_i - z_{i-1}|}
\label{eq:curvature}
\end{equation}

\subsubsection{Topological Measures}

The tortuosity of a trajectory quantifies its complexity:
\begin{equation}
\tau = \frac{\sum_{i=1}^{N-1} |z_{i+1} - z_i|}{|z_N - z_1|}
\label{eq:tortuosity}
\end{equation}

The winding number captures the trajectory's rotational behavior:
\begin{equation}
W = \frac{1}{2\pi} \sum_{i=1}^{N-1} \arg\left(\frac{z_{i+1} - z_c}{z_i - z_c}\right)
\label{eq:winding}
\end{equation}

where $z_c$ is the trajectory centroid.

\subsection{Spectral Analysis}

The Fourier transform of the complex trajectory provides frequency domain insights:

\begin{equation}
\tilde{Z}(\omega) = \int_{-\infty}^{\infty} z(t) e^{-i\omega t} dt
\label{eq:fourier}
\end{equation}

The spectral entropy quantifies the frequency domain complexity:

\begin{equation}
S = -\sum_k P_k \log_2 P_k
\label{eq:spectral_entropy}
\end{equation}

where $P_k = |\tilde{Z}(\omega_k)|^2 / \sum_j |\tilde{Z}(\omega_j)|^2$ is the normalized power spectral density.

\subsection{Important Distinction: MQC vs. Quantum Amplitudes}

It is crucial to emphasize that MQC numbers are \emph{not} quantum mechanical amplitudes. While both are complex numbers, they serve fundamentally different purposes:

\begin{itemize}
\item \textbf{Quantum amplitudes}: Describe the probability amplitudes of quantum states, follow superposition principles, and exhibit quantum interference
\item \textbf{MQC numbers}: Statistical indicators derived from measurement sequences, providing diagnostic information about measurement quality patterns
\end{itemize}

This distinction is essential for proper interpretation of MQC analysis results as practical diagnostic tools rather than fundamental quantum mechanical quantities.

\subsection{Error Detection Framework}

MQC-based error detection compares trajectory statistics against reference distributions. For a test trajectory $\mathcal{T}_{\text{test}}$ and reference trajectory $\mathcal{T}_{\text{ref}}$, we compute:

\begin{equation}
E = \sqrt{\sum_{p} w_p \left(\frac{S_p^{\text{test}} - S_p^{\text{ref}}}{S_p^{\text{ref}}}\right)^2}
\label{eq:error_metric}
\end{equation}

where $S_p$ represents various statistical measures (mean, variance, tortuosity, spectral entropy) and $w_p$ are weighting factors.

An error is detected when $E$ exceeds a threshold $E_{\text{th}}$ determined from clean reference data.
\section{Methods}
\label{sec:methods}

\subsection{Experimental Data}

Our analysis employs two distinct categories of quantum measurement data to validate MQC analysis as a practical diagnostic tool:

\subsubsection{Bell State Data}
We utilize measurement data from four maximally entangled Bell states:
\begin{align}
|\Phi^+\rangle &= \frac{1}{\sqrt{2}}(|00\rangle + |11\rangle) \\
|\Phi^-\rangle &= \frac{1}{\sqrt{2}}(|00\rangle - |11\rangle) \\
|\Psi^+\rangle &= \frac{1}{\sqrt{2}}(|01\rangle + |10\rangle) \\
|\Psi^-\rangle &= \frac{1}{\sqrt{2}}(|01\rangle - |10\rangle)
\end{align}

Each Bell state was measured 8,192 times, providing high-statistics data for analysis. These states serve as clean reference data due to their theoretical purity and well-characterized properties.

\subsubsection{IBM Quantum Volume Data}
We analyze real experimental data from IBM Quantum Volume (QV) experiments, which represent realistic noisy quantum computations. The dataset includes:
\begin{itemize}
\item Standard QV data: 70 experimental trials
\item Moderate noise QV data: 100 trials with intermediate noise levels  
\item High noise QV data: Variable trials with elevated noise conditions
\end{itemize}

These datasets provide realistic examples of quantum computations affected by various noise sources including decoherence, gate errors, and measurement errors.

\subsection{Trajectory Generation}

For each dataset, we generate complex trajectories using the MQC representation described in Section~\ref{sec:theory}. The specific implementation details are:

\subsubsection{Preprocessing}
\begin{enumerate}
\item Raw measurement counts are converted to binary sequences
\item Sequences are subsampled to ensure computational efficiency while preserving statistical properties
\item Window sizes are chosen adaptively based on sequence length: $w = \min(50, N/10)$
\end{enumerate}

\subsubsection{Complex Mapping}
Each binary measurement sequence $\{m_i\}$ is transformed into a complex trajectory $\{z_i\}$ using Equations~\ref{eq:direction} and \ref{eq:uncertainty}, with physical constraints applied according to Equations~\ref{eq:constraint_real} and \ref{eq:constraint_imag}.

\subsection{Analysis Pipeline}

Our comprehensive analysis pipeline consists of four main components:

\subsubsection{Geometric Analysis}
For each trajectory, we compute:
\begin{itemize}
\item Instantaneous properties: velocity (Eq.~\ref{eq:velocity}), acceleration (Eq.~\ref{eq:acceleration}), curvature (Eq.~\ref{eq:curvature})
\item Global properties: tortuosity (Eq.~\ref{eq:tortuosity}), total path length, trajectory area
\item Topological properties: winding number (Eq.~\ref{eq:winding}), self-intersection count
\end{itemize}

\subsubsection{W-Pattern Feature Extraction}
We implement specialized algorithms to detect and quantify W-shaped patterns in trajectories:
\begin{itemize}
\item Fractal dimension estimation using box-counting methods
\item Detection of characteristic turning points and inflection points
\item Quantification of pattern regularity and symmetry
\end{itemize}

\subsubsection{Spectral Analysis}
Frequency domain analysis is performed using:
\begin{itemize}
\item Fast Fourier Transform (FFT) of complex trajectories
\item Power spectral density computation for real and imaginary components
\item Spectral entropy calculation (Eq.~\ref{eq:spectral_entropy})
\item Time-frequency analysis using spectrograms
\end{itemize}

\subsubsection{Statistical Comparison}
We implement robust statistical methods for comparing trajectory properties:
\begin{itemize}
\item Kolmogorov-Smirnov tests for distribution comparisons
\item Bootstrap resampling for confidence interval estimation
\item Multi-dimensional scaling for trajectory clustering
\end{itemize}

\subsection{Error Detection Implementation}

The error detection system operates in two phases, incorporating specialized detection algorithms for different error types based on trajectory pattern analysis:

\subsubsection{Training Phase}
\begin{enumerate}
\item Reference statistics are computed from clean Bell state trajectories
\item Threshold values are determined using 3-sigma criteria on reference distributions
\item Feature weights $w_p$ in Equation~\ref{eq:error_metric} are optimized using cross-validation
\end{enumerate}

\subsubsection{Detection Phase}
\begin{enumerate}
\item Test trajectories are processed through the same analysis pipeline
\item Error metrics are computed using Equation~\ref{eq:error_metric}
\item Binary classification is performed based on threshold comparison
\item Confidence scores are assigned based on the magnitude of deviation from reference
\end{enumerate}

\subsubsection{Specialized Error Type Detection}
Building on trajectory pattern recognition, we implement targeted detection algorithms:

\begin{itemize}
\item \textbf{Bit-flip Error Detection}: Monitoring trajectory curvature spikes and sudden direction changes
\item \textbf{Phase Noise Detection}: Analyzing trajectory wandering patterns and phase drift
\item \textbf{Amplitude Decay Detection}: Tracking systematic drift toward the complex plane origin
\item \textbf{Coherence Loss Detection}: Measuring trajectory regularity and pattern degradation
\end{itemize}

Each detection algorithm uses error-specific thresholds optimized for maximum sensitivity while maintaining low false positive rates.

\subsection{Computational Implementation}

All analyses are implemented in Python 3.8+ using:
\begin{itemize}
\item NumPy and SciPy for numerical computations
\item Matplotlib for visualization
\item Custom algorithms for MQC-specific calculations
\end{itemize}

The complete analysis pipeline processes datasets containing thousands of measurements within minutes on standard desktop hardware, making it suitable for real-time quantum device monitoring applications.

\subsection{Validation Methodology}

To ensure robustness of our results, we employ:
\begin{itemize}
\item Cross-validation using different subsets of data
\item Sensitivity analysis for parameter variations
\item Comparison with traditional quantum state analysis methods
\item Statistical significance testing for all reported differences
\end{itemize}
\section{Results}
\label{sec:results}

\subsection{Trajectory Visualization and Basic Properties}

Figure~\ref{fig:trajectories} shows representative complex trajectories generated from Bell state and Quantum Volume data using MQC representation. The Bell state trajectories exhibit regular, predictable patterns with smooth curves and limited complexity, while the Quantum Volume trajectories display irregular, highly complex patterns with numerous direction changes and self-intersections.

\begin{figure}[htb]
\centering
\includegraphics[width=0.5\textwidth]{figures/publication_complex_trajectories.png}
\caption{Complex trajectories generated from real quantum measurement data. Top panels: Bell state data showing regular, smooth patterns with consistent statistical properties. Bottom panels: IBM Quantum Volume data exhibiting complex, irregular trajectories with higher variability. Green circles indicate starting points, red stars indicate ending points. Statistical information boxes show trajectory length, mean magnitude, and standard deviation.}
\label{fig:trajectories}
\end{figure}

Table~\ref{tab:basic_stats} summarizes the basic statistical properties of the generated trajectories. The Quantum Volume data consistently shows higher mean velocities (4.4-fold increase) and more variable trajectory lengths, indicating the effect of noise on trajectory dynamics.

\begin{table}[htb]
\centering
\caption{Basic trajectory statistics for different data types.}
\label{tab:basic_stats}
\begin{tabular}{lccc}
\hline\hline
Data Type & Trajectory Length & Mean Velocity & Max Acceleration \\
\hline
Bell States & 200--512 & 0.019 & 0.48 \\
Quantum Volume & 762--1600 & 0.084 & 0.49 \\
\hline\hline
\end{tabular}
\end{table}

\subsection{Geometric Analysis Results}

\subsubsection{Tortuosity Analysis}
MQC trajectory analysis reveals quantitative discrimination capability through tortuosity measurements. Bell state trajectories exhibit low tortuosity values (mean = 1.84), indicating nearly straight-line paths in the complex plane. In contrast, Quantum Volume trajectories show higher tortuosity (mean = 150.77), representing an 82-fold difference.

This quantitative difference in tortuosity provides a practical metric for distinguishing between clean and noisy quantum measurement data. Figure~\ref{fig:w_pattern} illustrates the detailed geometric analysis of trajectory features.

\begin{figure}[htb]
\centering
\includegraphics[width=0.5\textwidth]{figures/w_pattern_characteristics_comparison.png}
\caption{Comparative analysis of W-pattern characteristics. The 82-fold difference in tortuosity between Bell states and Quantum Volume data is clearly visible in panel (a). Panel (b) shows path length distributions, while panels (c) and (d) demonstrate the relationship between trajectory speed, complexity, and tortuosity.}
\label{fig:w_pattern}
\end{figure}

\subsubsection{Path Length Analysis}
The path length distributions show clear separation between data types:
\begin{itemize}
\item Bell states: Shorter, more consistent path lengths reflecting regular trajectories
\item Quantum Volume: Longer, more variable path lengths indicating complex wandering patterns
\end{itemize}

These differences in path length correlate with the tortuosity measurements, providing additional geometric discrimination between clean and noisy quantum measurements.

\subsection{Error Detection Performance}

The MQC-based error detection system achieves practical discrimination performance for distinguishing between clean and noisy quantum data:

\begin{table}[htb]
\centering
\caption{Error detection results for different quantum data types.}
\label{tab:error_detection}
\begin{tabular}{lc}
\hline\hline
Data Source & Error Detection Rate (\%) \\
\hline
Bell $\Phi^-$ & 0.0 \\
Bell $\Psi^+$ & 0.0 \\
Bell $\Psi^-$ & 0.0 \\
QV Clean & 8.5 $\pm$ 2.1 \\
QV Moderate & 15.3 $\pm$ 3.2 \\
QV Noisy & 26.4 $\pm$ 4.8 \\
\hline\hline
\end{tabular}
\end{table}

The results demonstrate practical utility for quantum device diagnostics:
\begin{itemize}
\item \textbf{Reliable specificity}: 0\% false positive rate for Bell states
\item \textbf{Noise-dependent sensitivity}: Error detection rates increase with noise level (8.5\% for QV Clean, 15.3\% for QV Moderate, 26.4\% for QV Noisy)
\item \textbf{Clear discrimination}: Monotonic increase in error detection with increasing noise levels
\end{itemize}

Figure~\ref{fig:error_detection} visualizes the error detection performance across different data types with statistical error bars.

\begin{figure}[htb]
\centering
\includegraphics[width=0.5\textwidth]{figures/improved_error_detection_results.png}
\caption{MQC error detection performance by data type. The bar plot shows clear discrimination between Bell states (0\% false positive rate) and Quantum Volume data with monotonically increasing error detection rates corresponding to noise levels. Error bars represent standard errors across multiple measurements.}
\label{fig:error_detection}
\end{figure}


\subsection{Spectral Analysis Results}

Fourier analysis reveals significant differences in frequency domain characteristics between clean and noisy data:

\subsubsection{Spectral Entropy}
The spectral entropy comparison shows a 3.6-fold difference:
\begin{itemize}
\item Bell states: 1.3463 (identical for all four states)
\item Quantum Volume: 4.92--5.01 (varying with noise level)
\end{itemize}

This quantitative difference in spectral entropy indicates that noisy data contains more frequency components, reflecting the broadband nature of quantum noise and providing another discrimination metric for MQC analysis.

\subsubsection{Frequency Domain Characteristics}
Table~\ref{tab:spectral_analysis} summarizes key spectral properties:

\begin{table}[htb]
\centering
\caption{Spectral analysis results showing frequency domain characteristics.}
\label{tab:spectral_analysis}
\begin{tabular}{lcc}
\hline\hline
Property & Bell States & Quantum Volume \\
\hline
Spectral Entropy & 1.35 & 4.97 \\
Mean Frequency (Hz) & 0.0049 & 0.0141 \\
Spectral Bandwidth (Hz) & 0.0179 & 0.0218 \\
Dominant Frequencies & Regular pattern & Irregular \\
\hline\hline
\end{tabular}
\end{table}

Figure~\ref{fig:spectral_comparison} shows the comprehensive spectral analysis comparison, highlighting the clear distinction between data types across multiple frequency domain metrics.

\begin{figure}[htb]
\centering
\includegraphics[width=0.5\textwidth]{figures/spectral_characteristics_comparison.png}
\caption{Comprehensive spectral characteristics comparison. Six different analysis perspectives demonstrate consistent separation between Bell states (clean) and Quantum Volume (noisy) data. The 3.6-fold difference in spectral entropy is clearly visible in the entropy vs. bandwidth plot (top-left).}
\label{fig:spectral_comparison}
\end{figure}


\subsection{Statistical Significance}

All reported differences between Bell state and Quantum Volume data are statistically significant with p-values < 0.001. The large effect sizes (82-fold tortuosity difference, 3.6-fold spectral entropy difference) provide robust discrimination capability that far exceeds typical statistical significance thresholds.

The consistency of results across multiple independent analytical approaches (geometric, spectral, topological, statistical) demonstrates the reliability of the MQC framework as a complementary tool for quantum data analysis.
\section{Discussion}
\label{sec:discussion}

\subsection{Practical Utility of MQC Analysis}

Our results demonstrate that MQC analysis provides a complementary tool for NISQ device characterization, particularly useful for real-time monitoring applications where computational efficiency is crucial. The ability to distinguish between clean Bell states and noisy Quantum Volume data (0\% false positive rate, with detection rates increasing monotonically from 8.5\% to 26.4\% with noise level) shows practical discrimination capability suitable for hardware diagnostic applications.

\subsubsection{Quantitative Discrimination Metrics}
The 82-fold difference in tortuosity between clean and noisy data provides a quantitative metric for assessing measurement quality. This sensitivity enables detection of performance degradation in quantum devices through visual trajectory analysis, complementing existing diagnostic methods.

The 3.6-fold difference in spectral entropy offers additional discrimination capability through frequency domain analysis. This multi-dimensional approach provides hardware engineers with intuitive visualization tools for identifying quantum device performance issues.

\subsection{Physical Interpretation}

\subsubsection{Geometric Meaning of Trajectories}
The complex trajectories generated by MQC analysis have practical interpretations for device diagnostics. The real component represents the directional bias of measurements, which correlates with measurement consistency. Clean Bell states maintain consistent measurement patterns, resulting in smooth trajectories with low tortuosity.

The imaginary component captures measurement uncertainty, which increases in the presence of noise. Noisy quantum systems exhibit higher uncertainty and more erratic measurement patterns, leading to complex trajectories with high tortuosity and irregular geometric properties.

\subsubsection{Spectral Signatures of Quantum Noise}
The spectral analysis reveals that quantum noise manifests as broadband frequency content in the MQC trajectory representation. Clean quantum measurements produce discrete, well-defined frequency components, while noisy measurements generate continuous spectral distributions with higher entropy.

This observation suggests that different types of quantum noise may have characteristic spectral signatures that could be identified using MQC analysis. Future work could develop diagnostic protocols based on spectral fingerprints for specific error types.

\subsection{Comparison with Existing Methods}

Traditional quantum error detection methods typically rely on:
\begin{itemize}
\item Quantum process tomography: Requires extensive measurements and computational resources
\item Fidelity calculations: Provide limited information about error types and sources
\item Randomized benchmarking: Gives average error rates but lacks real-time capability
\end{itemize}

MQC analysis offers complementary capabilities to these approaches:

\subsubsection{Computational Efficiency}
MQC analysis offers excellent computational speed, suitable for real-time applications. Unlike QPT which requires extensive computational resources over hours to days, MQC algorithms process measurements in milliseconds with linear scaling, enabling continuous monitoring with minimal overhead.

\subsubsection{Information Richness}
MQC analysis provides rich diagnostic information by encoding both directional and uncertainty components in complex trajectories. While not as comprehensive as full QPT, it preserves substantially more information than simple binary approaches like RB, enabling detailed pattern-based diagnostics.

\subsubsection{Interpretability}
The visual nature of complex trajectory analysis offers intuitive insights into quantum device behavior. This interpretability advantage over statistical methods facilitates rapid understanding of performance patterns and guides diagnostic decision-making.

Figure~\ref{fig:method_comparison} provides a qualitative comparison of MQC analysis with traditional quantum diagnostic methods across four key performance dimensions.

\begin{figure}[htb]
\centering
\includegraphics[width=0.45\textwidth]{figures/method_comparison.png}
\caption{Qualitative comparison of quantum device diagnostic methods. MQC analysis offers a practical balance between diagnostic capability and computational efficiency, enabling real-time monitoring while preserving substantial information content. QPT provides comprehensive accuracy but requires extensive computational resources, while RB offers fast average assessments with limited diagnostic detail.}
\label{fig:method_comparison}
\end{figure}

\subsection{Applications and Implications}

\subsubsection{Real-Time Quantum Device Monitoring}
The computational efficiency and discrimination capability of MQC analysis make it suitable for real-time quantum device monitoring applications. The method could be integrated into quantum control systems to provide continuous visual feedback about device performance trends.

\subsubsection{Hardware Characterization}
The ability to distinguish between different noise levels suggests that MQC analysis could serve as a complementary tool for quantum hardware characterization. Different quantum devices could be compared based on their trajectory signatures, providing visual diagnostic information.

\subsubsection{Quantum Algorithm Diagnostics}
Understanding the trajectory patterns associated with different quantum algorithms could guide diagnostic efforts. Algorithms that produce more regular trajectory patterns may indicate better implementation or more suitable hardware conditions.

\subsubsection{Performance Trend Analysis}
The temporal nature of trajectory analysis enables monitoring of performance trends over time. By tracking trajectory pattern changes, it may be possible to identify gradual performance degradation or predict maintenance needs.

\subsection{Limitations and Future Directions}

\subsubsection{Current Limitations}
While the results demonstrate practical utility, several limitations should be acknowledged:

\begin{itemize}
\item \textbf{Limited data diversity}: The current analysis focuses on Bell states and Quantum Volume data. Validation with a broader range of quantum states and algorithms is needed.
\item \textbf{Scalability}: The analysis has been demonstrated primarily on 2-4 qubit systems. Extension to larger quantum systems requires further investigation.
\item \textbf{Noise model dependence}: The effectiveness of the method may depend on the specific types of noise present in the quantum system.
\end{itemize}

\subsubsection{Future Research Directions}
Several promising directions for future research emerge from this work:

\begin{enumerate}
\item \textbf{Multi-qubit extension}: Developing MQC representations for larger quantum systems and investigating scalability challenges.

\item \textbf{Machine learning integration}: Combining MQC features with machine learning algorithms to develop more sophisticated diagnostic and classification systems.

\item \textbf{Real-time implementation}: Integrating MQC analysis into quantum control hardware for continuous monitoring and feedback applications.

\item \textbf{Diagnostic protocols}: Developing standardized diagnostic protocols specifically designed to work with MQC-based device characterization.

\item \textbf{Noise characterization}: Using MQC spectral signatures to classify and characterize different types of quantum noise for diagnostic purposes.
\end{enumerate}

\subsection{Broader Impact}

The development of MQC analysis represents a practical contribution to quantum device diagnostic capabilities. As NISQ-era quantum computing technology continues to develop, the need for accessible monitoring and characterization tools becomes increasingly important.

The ability to provide visual, intuitive diagnostics for quantum device performance could assist hardware engineers and algorithm developers in optimizing quantum systems. By providing complementary diagnostic information, MQC analysis contributes to the practical toolbox available for quantum device characterization.

The computational efficiency and visual interpretability of the approach make it accessible to practitioners who may not have extensive quantum error correction expertise, potentially broadening the community of researchers who can effectively characterize quantum device performance.
\section{Conclusion}
\label{sec:conclusion}

We have developed and validated Measurement Quality Complex (MQC) analysis, a practical tool for analyzing quantum measurement data through complex number representations. Our comprehensive analysis of real quantum data from Bell states and IBM Quantum Volume experiments demonstrates the utility of this approach for quantum device diagnostics and system characterization.

\subsection{Key Achievements}

The major achievements of this work include:

\begin{enumerate}
\item \textbf{Practical diagnostic framework}: Development of a computationally efficient framework for mapping quantum measurements to complex trajectories, providing both geometric and spectral visualization capabilities.

\item \textbf{Effective discrimination performance}: Achievement of reliable separation between clean and noisy quantum data, with 0\% false positive rates and noise-dependent detection rates increasing from 8.5\% to 26.4\%.

\item \textbf{Quantitative discrimination metrics}: Demonstration of substantial quantitative differences including 82-fold tortuosity variation and 3.6-fold spectral entropy differences between clean and noisy data.

\item \textbf{Multi-dimensional validation}: Confirmation of MQC effectiveness across multiple analytical approaches including geometric, spectral, topological, and statistical methods.

\item \textbf{Real-time capability}: Implementation of algorithms suitable for real-time device monitoring with linear scaling in the number of measurements.
\end{enumerate}

\subsection{Scientific Contributions}

This work makes several practical contributions to quantum device diagnostics:

\subsubsection{Methodological Innovation}
The MQC framework provides a complementary approach to traditional binary measurement analysis through continuous complex trajectory visualization. This approach preserves pattern information from quantum measurements while maintaining computational efficiency.

\subsubsection{Enhanced Diagnostic Capability}
The demonstrated discrimination capability of MQC-based analysis provides quantitative metrics for quantum device characterization, complementing existing diagnostic methods with visual and intuitive feedback.

\subsubsection{Practical Insights}
The geometric and spectral properties of complex trajectories provide practical insights into quantum measurement quality patterns and their relationship to device performance characteristics.

\subsection{Practical Impact}

The practical implications of this work include:

\begin{itemize}
\item \textbf{Quantum hardware diagnostics}: MQC analysis provides complementary metrics for characterizing and comparing quantum device performance.

\item \textbf{Real-time monitoring}: The computational efficiency enables continuous monitoring of quantum devices during operation with visual feedback.

\item \textbf{Performance trending}: Trajectory analysis enables tracking of device performance patterns over time, potentially identifying gradual degradation.

\item \textbf{Diagnostic toolbox}: MQC methods can serve as accessible diagnostic tools for practitioners working with NISQ-era quantum devices.
\end{itemize}

\subsection{Future Outlook}

The validation of MQC analysis with real quantum data opens several avenues for future development:

\subsubsection{Immediate Extensions}
\begin{itemize}
\item Validation with larger quantum systems (10+ qubits) and more diverse quantum algorithms
\item Integration with existing quantum device characterization workflows
\item Development of standardized MQC diagnostic protocols for different types of quantum hardware
\end{itemize}

\subsubsection{Long-term Directions}
\begin{itemize}
\item Machine learning integration for automated diagnostic classification and trend analysis
\item Extension to larger quantum systems while maintaining computational efficiency
\item Application to specialized quantum applications requiring real-time performance monitoring
\end{itemize}

\subsection{Closing Remarks}

The development and validation of MQC analysis provides a practical contribution to quantum device diagnostic capabilities. By demonstrating that quantum measurements can be effectively analyzed as complex trajectories with quantitative discrimination capability, this work provides complementary tools for the NISQ-era quantum computing community.

The 82-fold discrimination metric, combined with reliable separation between clean and noisy data, establishes MQC analysis as a useful complementary approach for quantum device diagnostics. As quantum computing continues to advance toward practical applications, accessible diagnostic tools like MQC analysis may become increasingly valuable for device characterization and performance monitoring.

The framework developed here addresses practical needs in quantum device diagnostics while maintaining computational efficiency suitable for real-time applications. We anticipate that MQC analysis will find applications as a complementary diagnostic tool and contribute to the practical toolbox available for quantum device characterization.

This work demonstrates that by viewing quantum measurements through complex trajectory visualization, we can provide intuitive and quantitative diagnostic information that complements traditional characterization approaches. The validation of MQC analysis with real quantum data supports this approach and provides a foundation for future developments in accessible quantum device diagnostic tools.

\textbf{Implementation Availability}: To support reproducibility and facilitate adoption by the quantum computing community, we have made the complete MQC analysis framework freely available as open-source software~\cite{github_mqc_analysis}. The implementation includes all algorithms, validation experiments, and example data analysis presented in this work, enabling researchers to directly apply and extend these methods.

\begin{acknowledgments}
We acknowledge the assistance of Claude (Anthropic's AI assistant) in developing the computational framework for complex quantum trajectory analysis, particularly in data processing, statistical analysis, and visualization implementation. We thank the IBM Quantum team for providing access to quantum hardware data through the Quantum Volume experiments.
\end{acknowledgments}

\bibliographystyle{apsrev4-1}
\bibliography{references}

\end{document}