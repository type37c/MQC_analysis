\documentclass[a4paper,11pt]{article}
\usepackage[utf8]{inputenc}
\usepackage{CJKutf8}

% 基本パッケージ
\usepackage{graphicx}
\usepackage{amsmath}
\usepackage{amssymb}
\usepackage{bm}
\usepackage{color}
\usepackage{hyperref}
\usepackage{subcaption}
\usepackage{float}
\usepackage{url}

% ページ設定
\usepackage[top=25mm,bottom=25mm,left=25mm,right=25mm]{geometry}

% セクションフォーマット
\usepackage{titlesec}
\titleformat{\section}{\Large\bfseries}{\thesection.}{1em}{}
\titleformat{\subsection}{\large\bfseries}{\thesubsection}{1em}{}
\titleformat{\subsubsection}{\normalsize\bfseries}{\thesubsubsection}{1em}{}

% キャプション設定
\usepackage[labelfont=bf,font=small]{caption}

\begin{document}
\begin{CJK}{UTF8}{min}

\title{測定品質複素数(MQC)解析:\\NISQ時代の量子コンピューティングのための視覚化ツール}

\author{大谷 圭亮\\
\texttt{wacon316@gmail.com}}

\date{\today}

\maketitle

\begin{abstract}
本研究では、測定品質複素数(MQC)表現を用いた量子測定データの実用的な視覚化・解析ツールを提案する。我々の手法は、IBMのQuantum Volume実験において18.6\%のエラー検出率を達成し、リアルタイム監視に適した計算効率を維持している。MQCアプローチは、測定シーケンスを複素数 $z = d + ui$($d$は方向バイアス、$u$は不確定性)として表現し、量子デバイスの性能を直感的に視覚化する。ベル状態とIBM Quantum Volumeデータでのテストでは、クリーンなデータとノイジーな測定データ間で82倍のトルトゥオシティ差と3.6倍のスペクトルエントロピー変動による定量的識別を実証した。このツールは、NISQ時代の量子コンピューティング実務者に、ハードウェア特性評価とリアルタイム品質評価のための補完的診断手法を提供する。
\end{abstract}

\section{はじめに}

ノイジー中規模量子(NISQ)デバイスの時代において、量子測定の品質を評価し可視化する実用的なツールの必要性がますます高まっている。現在の量子コンピュータは、デコヒーレンス、ゲートエラー、測定ノイズなどの様々なエラー源の影響を受けており、これらが量子アルゴリズムの信頼性と精度を制限している。量子プロセストモグラフィーやランダム化ベンチマーキングなどの従来の診断手法は貴重な洞察を提供するが、しばしば計算集約的であり、リアルタイム監視には適していない。

本研究では、測定品質複素数(MQC)解析と呼ばれる補完的アプローチを提案する。この手法は、量子測定シーケンスを複素軌跡として表現し、測定品質パターンの効率的な視覚化と特性評価を可能にする。我々は、MQCを基本的な物理量ではなく、デバイス診断のための実用的な統計ツールとして位置づけることで、NISQ時代のハードウェア開発者と研究者の両方にとってアクセスしやすくしている。

我々のアプローチは、量子測定データにおける隠れたパターンを明らかにする複素数表現の能力に動機づけられている。各測定結果 $m \in \{0, 1\}$ を、方向情報(実部)と不確定性情報(虚部)の両方をエンコードする複素数にマッピングすることで、標準的なバイナリ表現よりも豊富な解析フレームワークを作成する。この表現により、幾何学的、トポロジカル、スペクトル解析技術を適用でき、測定品質に関する多次元的な洞察を提供する。

\section{測定品質複素数表現}

\subsection{定義と動機}

MQCを、基本的な物理量ではなく、量子測定品質を視覚化するための有用な統計表現として定義する。結果 $m_i \in \{0, 1\}$ を生成する量子測定のシーケンスに対して、以下の複素数を構築する:

\begin{equation}
z_i = d_i + i u_i
\label{eq:cqt_basic}
\end{equation}

ここで、$d_i$ は局所ウィンドウ内での方向バイアスを表し、$u_i$ は測定の不確定性を表す。

\subsection{量子振幅との区別}

MQC数値は測定データのパターンを表現するものであり、量子力学的振幅ではないことを強調することが重要である。この区別は結果の適切な解釈にとって極めて重要である。

\begin{table}[h]
\centering
\caption{量子振幅とMQC数値の基本的な違い}
\begin{tabular}{lll}
\hline
性質 & 量子振幅 & MQC \\
\hline
物理的意味 & 確率振幅 & 統計的指標 \\
正規化 & $\sum|\psi|^2 = 1$ & 不要 \\
干渉 & あり & なし \\
\hline
\end{tabular}
\end{table}

\subsection{方向と不確定性のマッピング}

方向成分 $d_i$ は実行中の相関測度として計算される:

\begin{equation}
d_i = \frac{1}{w} \sum_{j=i-w+1}^{i} \frac{2m_j - 1}{\sqrt{1 + \sigma^2_j}}
\label{eq:direction}
\end{equation}

ここで、$w$ は局所平均化のウィンドウサイズ、$\sigma^2_j$ は測定の局所分散である。

不確定性成分 $u_i$ は測定シーケンスの予測不可能性を捉える:

\begin{equation}
u_i = \sqrt{\frac{1}{w-1} \sum_{j=i-w+1}^{i} (m_j - \bar{m}_i)^2}
\label{eq:uncertainty}
\end{equation}

ここで、$\bar{m}_i$ はウィンドウ内の測定の局所平均である。

\subsection{軌跡構築と解析}

$N$ 個の測定のシーケンスは複素軌跡を生成する:

\begin{equation}
\mathcal{T} = \{z_1, z_2, \ldots, z_N\}
\label{eq:trajectory}
\end{equation}

この軌跡は、幾何学的性質(速度、加速度、曲率)、トポロジカル測度(トルトゥオシティ、巻き数)、スペクトル解析(フーリエ変換、スペクトルエントロピー)など、様々な数学的ツールを使用して解析できる。

\section{実験方法}

\subsection{実験データ}

我々の解析では2種類の量子測定データを使用した:

\subsubsection{ベル状態データ}
クリーンな量子測定の基準として、4つのベル状態($|\Phi^+\rangle$、$|\Phi^-\rangle$、$|\Psi^+\rangle$、$|\Psi^-\rangle$)から収集されたデータを使用した。これらの状態は、理想的な条件下で既知の統計的性質を示し、我々の手法を検証するための基準を提供する。

\subsubsection{IBM Quantum Volumeデータ}
実世界のノイジーな量子測定を表すために、IBMのQuantum Volume実験からのデータを解析した。このデータセットには、量子デバイスの現実的な性能特性を反映した様々なノイズレベルの測定が含まれている。

\subsection{軌跡生成パイプライン}

測定データは以下のパイプラインを通じて処理される:

1. **前処理**:生の測定カウントを正規化し、時系列にフォーマット
2. **複素マッピング**:式(\ref{eq:direction})と(\ref{eq:uncertainty})を使用して各測定を複素数に変換
3. **軌跡構築**:複素数のシーケンスから連続軌跡を形成
4. **特徴抽出**:幾何学的、トポロジカル、スペクトル特性を計算

\subsection{解析手法}

\subsubsection{幾何学的解析}
瞬間速度、加速度、曲率を計算して軌跡のダイナミクスを特性化した。これらの測度は、測定パターンが時間とともにどのように進化するかについての洞察を提供する。

\subsubsection{W-パターン特徴抽出}
軌跡の複雑さを定量化するためにトルトゥオシティ(経路長と直線距離の比)を計算した。また、ボックスカウント法を使用してフラクタル次元を推定した。

\subsubsection{スペクトル解析}
フーリエ変換を適用して周波数領域特性を抽出した。スペクトルエントロピーは、周波数成分の分布を定量化し、ノイズの複雑さの測度を提供する。

\section{結果}

\subsection{軌跡の視覚化と基本特性}

図1は、ベル状態とQuantum Volumeデータから生成された代表的な複素軌跡を示している。ベル状態の軌跡は、限られた複雑さで規則的で予測可能なパターンを示す一方、Quantum Volumeの軌跡は、多数の方向変化と自己交差を持つ不規則で高度に複雑なパターンを表示する。

\begin{figure}[H]
\centering
\includegraphics[width=0.8\textwidth]{figures/publication_complex_trajectories.png}
\caption{実際の量子測定データから生成された複素軌跡。上段:一貫した統計的性質を持つ規則的で滑らかなパターンを示すベル状態データ。下段:高い変動性を持つ複雑で不規則な軌跡を示すIBM Quantum Volumeデータ。}
\label{fig:trajectories}
\end{figure}

\subsection{幾何学的解析結果}

\subsubsection{トルトゥオシティ解析}
MQC軌跡解析は、トルトゥオシティ測定による定量的識別能力を明らかにする。ベル状態の軌跡は低いトルトゥオシティ値(平均 = 1.84)を示し、複素平面でほぼ直線的な経路を示す。対照的に、Quantum Volumeの軌跡は高いトルトゥオシティ(平均 = 150.77)を示し、82倍の差を表している。

\begin{figure}[H]
\centering
\includegraphics[width=0.8\textwidth]{figures/w_pattern_characteristics_comparison.png}
\caption{W-パターン特性の比較分析。ベル状態とQuantum Volumeデータ間の82倍のトルトゥオシティ差が左下パネルに明確に表示されている。}
\label{fig:w_pattern}
\end{figure}

\subsection{エラー検出性能}

MQCベースのエラー検出システムは、クリーンなデータとノイジーな量子データを区別するための実用的な識別性能を達成する:

\begin{table}[h]
\centering
\caption{異なる量子データタイプのエラー検出結果}
\begin{tabular}{lccc}
\hline
データソース & エラー率 & 平均重大度 & 最大重大度 \\
\hline
Bell $\Phi^-$ & 0.0000 & 0.0000 & 0.0000 \\
Bell $\Psi^+$ & 0.0000 & 0.0000 & 0.0000 \\
Bell $\Psi^-$ & 0.0000 & 0.0000 & 0.0000 \\
QV Clean & 0.2634 & 1.1599 & 2.0000 \\
QV Moderate & 0.1349 & 1.2898 & 2.0000 \\
QV Noisy & 0.1604 & 1.1542 & 2.2146 \\
\hline
\end{tabular}
\end{table}

結果は量子デバイス診断のための実用的な有用性を示している:
- **信頼できる特異性**:ベル状態に対して0\%の偽陽性率
- **測定可能な感度**:Quantum Volumeデータに対して平均18.6\%のエラー検出率
- **明確な識別**:クリーンなデータとノイジーなデータの分類間の効果的な分離

\begin{figure}[H]
\centering
\includegraphics[width=0.8\textwidth]{figures/improved_error_detection_results.png}
\caption{データタイプ別のMQCエラー検出性能。ベル状態(0\%偽陽性率)とQuantum Volumeデータ(平均18.6\%検出率)間の明確な識別を示す。}
\label{fig:error_detection}
\end{figure}

\subsection{スペクトル解析結果}

フーリエ解析により、クリーンなデータとノイジーなデータ間の周波数領域特性に有意な差が明らかになった:

\subsubsection{スペクトルエントロピー}
スペクトルエントロピーの比較は3.6倍の差を示した:
- ベル状態:1.3463(4つの状態すべてで同一)
- Quantum Volume:4.92〜5.01(ノイズレベルによって変動)

この定量的なスペクトルエントロピーの差は、ノイジーなデータがより多くの周波数成分を含むことを示し、量子ノイズの広帯域性を反映し、MQC解析のための別の識別メトリクスを提供する。

\begin{figure}[H]
\centering
\includegraphics[width=0.8\textwidth]{figures/spectral_characteristics_comparison.png}
\caption{包括的なスペクトル特性の比較。6つの異なる解析視点が、ベル状態(クリーン)とQuantum Volume(ノイジー)データ間の一貫した分離を示す。}
\label{fig:spectral_comparison}
\end{figure}

\section{考察}

\subsection{MQC解析の実用的有用性}

我々の結果は、MQC解析がNISQデバイス特性評価のための補完的ツールを提供することを実証している。特に、計算効率が重要となるリアルタイム監視アプリケーションに有用である。クリーンなベル状態とノイジーなQuantum Volumeデータを区別する能力(0\%の偽陽性率、平均18.6\%の検出率)は、ハードウェア診断アプリケーションに適した実用的な識別能力を示している。

\subsection{既存手法との比較}

従来の量子エラー検出手法は通常以下に依存する:
- 量子プロセストモグラフィー:広範な測定と計算資源を必要とする
- フィデリティ計算:エラータイプと原因に関する限定的な情報を提供
- ランダム化ベンチマーキング:平均エラー率を提供するがリアルタイム機能に欠ける

MQC解析はこれらのアプローチに対して補完的な機能を提供する:

\begin{figure}[H]
\centering
\includegraphics[width=0.8\textwidth]{figures/method_comparison.png}
\caption{量子デバイス診断手法の包括的比較。MQC解析は、QPTよりもアクセスしやすく良好な精度を提供することを示す。}
\label{fig:method_comparison}
\end{figure}

\subsection{限界と今後の方向性}

結果は実用的な有用性を示しているが、いくつかの限界を認識すべきである:
- 限定的なデータの多様性:現在の解析はベル状態とQuantum Volumeデータに焦点を当てている
- スケーラビリティ:解析は主に2-4量子ビットシステムで実証されている
- ノイズモデル依存性:手法の有効性は量子システムに存在する特定のノイズタイプに依存する可能性がある

今後の研究の有望な方向性には以下が含まれる:
- より大規模な量子システムへのMQC表現の拡張
- MQC特徴と機械学習アルゴリズムの組み合わせ
- 連続監視のための量子制御ハードウェアへのMQC解析の統合

\section{結論}

本研究では、NISQ時代の量子コンピューティングのための実用的な視覚化および診断ツールとして測定品質複素数(MQC)解析を提案した。我々の主要な成果は以下の通りである:

\subsection{主要な成果}

- **効果的な識別性能**:クリーンなデータとノイジーな量子測定間の82倍のトルトゥオシティと3.6倍のスペクトルエントロピー差
- **計算効率**:リアルタイム監視アプリケーションに適した線形スケーリング
- **実用的な診断能力**:ベル状態に対して0\%の偽陽性率、ノイジーデータに対して18.6\%の検出率
- **多次元解析**:単一のフレームワーク内で幾何学的、トポロジカル、スペクトル特性を組み合わせる

\subsection{科学的貢献}

MQCフレームワークは、連続的な複素軌跡視覚化を通じて量子測定品質評価のための補完的アプローチを提供する。MQCを基本的な理論的構成要素ではなく実用的な診断ツールとして位置づけることで、我々のアプローチは幅広い実務者にとってアクセスしやすく、NISQ時代のハードウェア特性評価の特定のニーズに対応している。

\subsection{実用的影響}

MQC解析の計算効率と視覚的解釈可能性により、以下に適している:
- リアルタイム量子デバイス監視
- ハードウェア性能の傾向分析
- 量子アルゴリズム診断
- 品質管理とデバイス認証

NISQ時代の量子コンピューティング技術が発展し続ける中、アクセス可能な監視と特性評価ツールの必要性がますます重要になっている。MQC解析は、基本的な量子力学的記述に代わるものではなく、実用的な診断機能を提供することで、既存の手法を補完するものである。

\section*{謝辞}

複素量子軌跡解析の計算フレームワークの開発、特にデータ処理、統計解析、視覚化実装において、Claude(AnthropicのAIアシスタント)の支援に感謝する。また、Quantum Volume実験を通じて量子ハードウェアデータへのアクセスを提供してくれたIBM Quantumチームに感謝する。

\end{CJK}
\end{document}